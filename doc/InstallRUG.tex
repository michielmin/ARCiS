\documentclass[12pt]{article}

\usepackage{natbib}
\usepackage{amsmath}
\usepackage{amssymb}
\usepackage{graphicx}
\usepackage{txfonts}
\usepackage[english]{babel}
\usepackage{hyperref}
\usepackage{units}
\usepackage{listings}
\usepackage{xcolor}

\lstdefinelanguage{args}{
sensitive=false,
alsoletter={.},
moredelim=[s][\color{red}]{<}{>},
moredelim=[s][\color{blue}]{[}{]},
moredelim=[is][\color{orange}]{:}{:},
keywords=[10]{...},
keywordstyle=[10]{\color{magenta}},
}

\lstnewenvironment{arguments}
{\lstset{language=args}}
{}

\lstnewenvironment{bash}
{\lstset{numbers=left,language=bash,keywordstyle={\color{blue}}}}
{}

\newcommand{\shellcmd}[1]{\\ \\ \indent\indent\texttt{\# #1}\\ \\ }

\newcommand{\inputfile}[1]{\\ \indent\indent\texttt{#1}}

\hypersetup{colorlinks=true, linkcolor=blue, citecolor=blue, urlcolor=blue}

%%%%%%%%%%%%%%%%%%%%%%%%%%%%%%%%%%%%%%%%
%
% Bibliography and bibfile
\def\aj{AJ}%
          % Astronomical Journal
\def\araa{ARA\&A}%
          % Annual Review of Astron and Astrophys
\def\apj{ApJ}%
          % Astrophysical Journal
\def\apjl{ApJ}%
          % Astrophysical Journal, Letters
\def\apjs{ApJS}%
          % Astrophysical Journal, Supplement
\def\ao{Appl.~Opt.}%
          % Applied Optics
\def\apss{Ap\&SS}%
          % Astrophysics and Space Science
\def\aap{A\&A}%
          % Astronomy and Astrophysics
\def\aapr{A\&A~Rev.}%
          % Astronomy and Astrophysics Reviews
\def\aaps{A\&AS}%
          % Astronomy and Astrophysics, Supplement
\def\azh{AZh}%
          % Astronomicheskii Zhurnal
\def\baas{BAAS}%
          % Bulletin of the AAS
\def\icarus{Icarus}%
          % Icarus
\def\jrasc{JRASC}%
          % Journal of the RAS of Canada
\def\memras{MmRAS}%
          % Memoirs of the RAS
\def\mnras{MNRAS}%
          % Monthly Notices of the RAS
\def\pra{Phys.~Rev.~A}%
          % Physical Review A: General Physics
\def\prb{Phys.~Rev.~B}%
          % Physical Review B: Solid State
\def\prc{Phys.~Rev.~C}%
          % Physical Review C
\def\prd{Phys.~Rev.~D}%
          % Physical Review D
\def\pre{Phys.~Rev.~E}%
          % Physical Review E
\def\prl{Phys.~Rev.~Lett.}%
          % Physical Review Letters
\def\pasp{PASP}%
          % Publications of the ASP
\def\pasj{PASJ}%
          % Publications of the ASJ
\def\qjras{QJRAS}%
          % Quarterly Journal of the RAS
\def\skytel{S\&T}%
          % Sky and Telescope
\def\solphys{Sol.~Phys.}%
          % Solar Physics
\def\sovast{Soviet~Ast.}%
          % Soviet Astronomy
\def\ssr{Space~Sci.~Rev.}%
          % Space Science Reviews
\def\zap{ZAp}%
          % Zeitschrift fuer Astrophysik
\def\nat{Nature}%
          % Nature
\def\iaucirc{IAU~Circ.}%
          % IAU Cirulars
\def\aplett{Astrophys.~Lett.}%
          % Astrophysics Letters
\def\apspr{Astrophys.~Space~Phys.~Res.}%
          % Astrophysics Space Physics Research
\def\bain{Bull.~Astron.~Inst.~Netherlands}%
          % Bulletin Astronomical Institute of the Netherlands
\def\fcp{Fund.~Cosmic~Phys.}%
          % Fundamental Cosmic Physics
\def\gca{Geochim.~Cosmochim.~Acta}%
          % Geochimica Cosmochimica Acta
\def\grl{Geophys.~Res.~Lett.}%
          % Geophysics Research Letters
\def\jcp{J.~Chem.~Phys.}%
          % Journal of Chemical Physics
\def\jgr{J.~Geophys.~Res.}%
          % Journal of Geophysics Research
\def\jqsrt{J.~Quant.~Spec.~Radiat.~Transf.}%
          % Journal of Quantitiative Spectroscopy and Radiative Trasfer
\def\memsai{Mem.~Soc.~Astron.~Italiana}%
          % Mem. Societa Astronomica Italiana
\def\nphysa{Nucl.~Phys.~A}%
          % Nuclear Physics A
\def\physrep{Phys.~Rep.}%
          % Physics Reports
\def\physscr{Phys.~Scr}%
          % Physica Scripta
\def\planss{Planet.~Space~Sci.}%
          % Planetary Space Science
\def\procspie{Proc.~SPIE}%
          % Proceedings of the SPIE
\let\astap=\aap
\let\apjlett=\apjl
\let\apjsupp=\apjs
\let\applopt=\ao
%
%


\usepackage{natbib}
\bibliographystyle{abbrvnat}

\begin{document}

\title{\includegraphics[width=0.9\hsize]{ARCiS}\\Installation on the computers of the University of Groningen}
\author{Michiel Min}
\date{\today}
\maketitle

\section{Introduction}


\subsection{Terms of use}

By using ARCiS you agree to the following:
\begin{itemize}
\item If in doubt on any of the results, you consult with me. Email: M.Min@sron.nl
\item You cite the appropriate papers listed below.
\end{itemize}
The most important reason for this is to make sure that ARCiS is used in a correct way and the result are scientifically useful. ARCiS is a complex code which can do a lot of things, this also means things can go wrong. Please refer to \cite{2020A&A...642A..28M} for the first full description of the fundamental properties of the code.

Note that there are several parts of the code from different developers:
\begin{itemize}
\item Cloud formation framework: \cite{2019A&A...622A.121O}
\item Optical properties of cloud particles computed using DHS: \cite{2005A&A...432..909M, 1981ApOpt..20.3657T}
\item Refractive indices for the cloud species, see references in \cite{2020A&A...642A..28M}
\item Molecular opacities: \cite{2021A&A...646A..21C} and references therein
\item Multinest Retrieval tools: \cite{2008MNRAS.384..449F, 2009MNRAS.398.1601F, 2019OJAp....2E..10F}
\item GGchem when including chemistry: \cite{2018A&A...614A...1W}
\item Disequilibrium chemistry implementation: \cite{2021A&A...656A..90K}
\item Diffusion implementation for 3D structures: \cite{2022A&A...665A...2C}
\item Coupling with planet formation parameters: \cite{2022A&A...667A.147K}
\end{itemize}

\section{Installing ARCiS at the computers of Kapteyn}

ARCiS is installed on the computers of Kapteyn by the computer group. The version number can be found at the top of the log file when running ARCiS.
To setup your computer for running ARCiS you need to only link to the location of the Data directory. This can be done as follows:

First go to your home directory:
\shellcmd{cd}
Next, make the ARCiS directory:
\shellcmd{mkdir ARCiS}
Now link the data directory to the right location:
\shellcmd{cd ARCiS}
\vspace{-2cm}\\
\shellcmd{ln -s /dataserver/users/formingworlds/ARCiSData ./Data}

Now when you want to run ARCiS, all you have to do is:
\shellcmd{module load ARCiS}

and you are good to go!

\bibliography{refs} % Entries are in the refs.bib file

\end{document}

