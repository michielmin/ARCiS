\documentclass[12pt]{article}

\usepackage{natbib}
\usepackage{amsmath}
\usepackage{amssymb}
\usepackage{graphicx}
\usepackage{txfonts}
\usepackage[english]{babel}
\usepackage{hyperref}
\usepackage{units}
\usepackage{listings}
\usepackage{xcolor}

\lstdefinelanguage{args}{
sensitive=false,
alsoletter={.},
moredelim=[s][\color{red}]{<}{>},
moredelim=[s][\color{blue}]{[}{]},
moredelim=[is][\color{orange}]{:}{:},
keywords=[10]{...},
keywordstyle=[10]{\color{magenta}},
}

\lstnewenvironment{arguments}
{\lstset{language=args}}
{}

\lstnewenvironment{bash}
{\lstset{numbers=left,language=bash,keywordstyle={\color{blue}}}}
{}

\newcommand{\shellcmd}[1]{\\ \\ \indent\indent\texttt{\# #1}\\ \\ }

\newcommand{\inputfile}[1]{\\ \indent\indent\texttt{#1}}

\hypersetup{colorlinks=true, linkcolor=blue, citecolor=blue, urlcolor=blue}

%%%%%%%%%%%%%%%%%%%%%%%%%%%%%%%%%%%%%%%%
%
% Bibliography and bibfile
\def\aj{AJ}%
          % Astronomical Journal
\def\araa{ARA\&A}%
          % Annual Review of Astron and Astrophys
\def\apj{ApJ}%
          % Astrophysical Journal
\def\apjl{ApJ}%
          % Astrophysical Journal, Letters
\def\apjs{ApJS}%
          % Astrophysical Journal, Supplement
\def\ao{Appl.~Opt.}%
          % Applied Optics
\def\apss{Ap\&SS}%
          % Astrophysics and Space Science
\def\aap{A\&A}%
          % Astronomy and Astrophysics
\def\aapr{A\&A~Rev.}%
          % Astronomy and Astrophysics Reviews
\def\aaps{A\&AS}%
          % Astronomy and Astrophysics, Supplement
\def\azh{AZh}%
          % Astronomicheskii Zhurnal
\def\baas{BAAS}%
          % Bulletin of the AAS
\def\icarus{Icarus}%
          % Icarus
\def\jrasc{JRASC}%
          % Journal of the RAS of Canada
\def\memras{MmRAS}%
          % Memoirs of the RAS
\def\mnras{MNRAS}%
          % Monthly Notices of the RAS
\def\pra{Phys.~Rev.~A}%
          % Physical Review A: General Physics
\def\prb{Phys.~Rev.~B}%
          % Physical Review B: Solid State
\def\prc{Phys.~Rev.~C}%
          % Physical Review C
\def\prd{Phys.~Rev.~D}%
          % Physical Review D
\def\pre{Phys.~Rev.~E}%
          % Physical Review E
\def\prl{Phys.~Rev.~Lett.}%
          % Physical Review Letters
\def\pasp{PASP}%
          % Publications of the ASP
\def\pasj{PASJ}%
          % Publications of the ASJ
\def\qjras{QJRAS}%
          % Quarterly Journal of the RAS
\def\skytel{S\&T}%
          % Sky and Telescope
\def\solphys{Sol.~Phys.}%
          % Solar Physics
\def\sovast{Soviet~Ast.}%
          % Soviet Astronomy
\def\ssr{Space~Sci.~Rev.}%
          % Space Science Reviews
\def\zap{ZAp}%
          % Zeitschrift fuer Astrophysik
\def\nat{Nature}%
          % Nature
\def\iaucirc{IAU~Circ.}%
          % IAU Cirulars
\def\aplett{Astrophys.~Lett.}%
          % Astrophysics Letters
\def\apspr{Astrophys.~Space~Phys.~Res.}%
          % Astrophysics Space Physics Research
\def\bain{Bull.~Astron.~Inst.~Netherlands}%
          % Bulletin Astronomical Institute of the Netherlands
\def\fcp{Fund.~Cosmic~Phys.}%
          % Fundamental Cosmic Physics
\def\gca{Geochim.~Cosmochim.~Acta}%
          % Geochimica Cosmochimica Acta
\def\grl{Geophys.~Res.~Lett.}%
          % Geophysics Research Letters
\def\jcp{J.~Chem.~Phys.}%
          % Journal of Chemical Physics
\def\jgr{J.~Geophys.~Res.}%
          % Journal of Geophysics Research
\def\jqsrt{J.~Quant.~Spec.~Radiat.~Transf.}%
          % Journal of Quantitiative Spectroscopy and Radiative Trasfer
\def\memsai{Mem.~Soc.~Astron.~Italiana}%
          % Mem. Societa Astronomica Italiana
\def\nphysa{Nucl.~Phys.~A}%
          % Nuclear Physics A
\def\physrep{Phys.~Rep.}%
          % Physics Reports
\def\physscr{Phys.~Scr}%
          % Physica Scripta
\def\planss{Planet.~Space~Sci.}%
          % Planetary Space Science
\def\procspie{Proc.~SPIE}%
          % Proceedings of the SPIE
\let\astap=\aap
\let\apjlett=\apjl
\let\apjsupp=\apjs
\let\applopt=\ao
%
%



\begin{document}

\title{\includegraphics[width=0.9\hsize]{ARCiS}\\Albedo iteration in 3D setups}
\author{Michiel Min}
\date{\today}
\maketitle

\section{Introduction}

In this document the albedo iterations are explained that make sure that also with a varying albedo between day and nightside of the planet, the total energy is conserved. The problem is here that the 3D scheme in principle uses a $\beta$-map that is not aware of if the local PT point is on the day side or on the night side. This creates a problem when the night side has a different reflectivity as the day side as the night side cannot reflect stellar radiation.

\subsection{Equations}

As explained in the WASP-43b paper (Chubb \& Min), the $\beta$-map represents a way of redistributing heat from the day to the night side. We have first the static $\beta$-map:
\begin{equation}
\beta_\star=\cos\Lambda\cos\Phi,
\end{equation}
on the dayside and
\begin{equation}
\beta_\star=0,
\end{equation}
on the nightside. Here $\Lambda$ and $\Phi$ are the longitude and latitude respectively.
The $\beta$ map is computed from a diffusion equation using $\beta_\star$ as source term. Note that $\beta$ and $\beta_\star$ have the property that:
\begin{equation}
\label{eq:normalise}
\int_\mathrm{planet}\beta=\int_\mathrm{planet}\beta_\star=\int_\mathrm{day}\beta_\star=1
\end{equation}

The total amount of reflected light from the planet is:
\begin{equation}
f_\mathrm{ref}=\int_\mathrm{day}\beta_\star\omega
\end{equation}

If the albedo, $\omega$, is a constant over the day and the night side, the fraction of starlight reflected taken into the computation of the P-T structure is:
\begin{equation}
f_\mathrm{ref}^{'}=\int_\mathrm{planet}\beta\omega=\omega \int_\mathrm{planet}\beta = \int_\mathrm{day}\beta_\star\omega =\omega
\end{equation}
This only works if the albedo is constant. If the albedo on the day and the night side is different $f_\mathrm{ref}^{'} \ne f_\mathrm{ref}$, which makes the energy balance incorrect.

To solve this issue we can add a scaling to the $\beta$-map. The total emission needs to be $1-f_\mathrm{ref}$. If we take the new $\beta$-map to be:
\begin{equation}
\beta^{'}=\gamma\beta
\end{equation}
with $\gamma$ a constant. We now have that the emission we use for the computation of the PT structure is equal to
\begin{equation}
\gamma-\int_\mathrm{planet}\gamma\beta(\omega)
\end{equation}
To have the right energy balance we need to make sure that
\begin{equation}
\gamma-\int_\mathrm{planet}\gamma\beta\omega = 1-\int_\mathrm{day}\beta_\star\omega
\end{equation}
which gives for the scaling factor
\begin{equation}
\gamma=\frac{1-\int_\mathrm{day}\beta_\star\omega}{1-\int_\mathrm{planet}\beta\omega}
\end{equation}
Note that for constant $\omega$ we can use Eq.~\ref{eq:normalise} to show that $\gamma=1$ (as expected).

To compute $\gamma$ we need to know the albedo at each location of the planet. For this we start with $\gamma=1$ and compute the albedo everywhere. We use this to compute a new estimate of $\gamma$. Usually this is already enough and the value of $\gamma$ does not change after one iteration. In the case of self-consistent cloud formation the albedo might depend more heavily on the PT structure and more than 1 iteration is required for $\gamma$ to converge.

\end{document}
