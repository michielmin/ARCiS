\documentclass[12pt]{article}

\usepackage{natbib}
\usepackage{amsmath}
\usepackage{amssymb}
\usepackage{graphicx}
\usepackage{txfonts}
\usepackage[english]{babel}
\usepackage{hyperref}
\usepackage{units}
\usepackage{listings}
\usepackage{xcolor}

\lstdefinelanguage{args}{
sensitive=false,
alsoletter={.},
moredelim=[s][\color{red}]{<}{>},
moredelim=[s][\color{blue}]{[}{]},
moredelim=[is][\color{orange}]{:}{:},
keywords=[10]{...},
keywordstyle=[10]{\color{magenta}},
}

\lstnewenvironment{arguments}
{\lstset{language=args}}
{}

\lstnewenvironment{bash}
{\lstset{numbers=left,language=bash,keywordstyle={\color{blue}}}}
{}

\newcommand{\shellcmd}[1]{\\ \\ \indent\indent\texttt{\# #1}\\ \\ }

\newcommand{\inputfile}[1]{\\ \indent\indent\texttt{#1}}

\hypersetup{colorlinks=true, linkcolor=blue, citecolor=blue, urlcolor=blue}

%%%%%%%%%%%%%%%%%%%%%%%%%%%%%%%%%%%%%%%%
%
% Bibliography and bibfile
\def\aj{AJ}%
          % Astronomical Journal
\def\araa{ARA\&A}%
          % Annual Review of Astron and Astrophys
\def\apj{ApJ}%
          % Astrophysical Journal
\def\apjl{ApJ}%
          % Astrophysical Journal, Letters
\def\apjs{ApJS}%
          % Astrophysical Journal, Supplement
\def\ao{Appl.~Opt.}%
          % Applied Optics
\def\apss{Ap\&SS}%
          % Astrophysics and Space Science
\def\aap{A\&A}%
          % Astronomy and Astrophysics
\def\aapr{A\&A~Rev.}%
          % Astronomy and Astrophysics Reviews
\def\aaps{A\&AS}%
          % Astronomy and Astrophysics, Supplement
\def\azh{AZh}%
          % Astronomicheskii Zhurnal
\def\baas{BAAS}%
          % Bulletin of the AAS
\def\icarus{Icarus}%
          % Icarus
\def\jrasc{JRASC}%
          % Journal of the RAS of Canada
\def\memras{MmRAS}%
          % Memoirs of the RAS
\def\mnras{MNRAS}%
          % Monthly Notices of the RAS
\def\pra{Phys.~Rev.~A}%
          % Physical Review A: General Physics
\def\prb{Phys.~Rev.~B}%
          % Physical Review B: Solid State
\def\prc{Phys.~Rev.~C}%
          % Physical Review C
\def\prd{Phys.~Rev.~D}%
          % Physical Review D
\def\pre{Phys.~Rev.~E}%
          % Physical Review E
\def\prl{Phys.~Rev.~Lett.}%
          % Physical Review Letters
\def\pasp{PASP}%
          % Publications of the ASP
\def\pasj{PASJ}%
          % Publications of the ASJ
\def\qjras{QJRAS}%
          % Quarterly Journal of the RAS
\def\skytel{S\&T}%
          % Sky and Telescope
\def\solphys{Sol.~Phys.}%
          % Solar Physics
\def\sovast{Soviet~Ast.}%
          % Soviet Astronomy
\def\ssr{Space~Sci.~Rev.}%
          % Space Science Reviews
\def\zap{ZAp}%
          % Zeitschrift fuer Astrophysik
\def\nat{Nature}%
          % Nature
\def\iaucirc{IAU~Circ.}%
          % IAU Cirulars
\def\aplett{Astrophys.~Lett.}%
          % Astrophysics Letters
\def\apspr{Astrophys.~Space~Phys.~Res.}%
          % Astrophysics Space Physics Research
\def\bain{Bull.~Astron.~Inst.~Netherlands}%
          % Bulletin Astronomical Institute of the Netherlands
\def\fcp{Fund.~Cosmic~Phys.}%
          % Fundamental Cosmic Physics
\def\gca{Geochim.~Cosmochim.~Acta}%
          % Geochimica Cosmochimica Acta
\def\grl{Geophys.~Res.~Lett.}%
          % Geophysics Research Letters
\def\jcp{J.~Chem.~Phys.}%
          % Journal of Chemical Physics
\def\jgr{J.~Geophys.~Res.}%
          % Journal of Geophysics Research
\def\jqsrt{J.~Quant.~Spec.~Radiat.~Transf.}%
          % Journal of Quantitiative Spectroscopy and Radiative Trasfer
\def\memsai{Mem.~Soc.~Astron.~Italiana}%
          % Mem. Societa Astronomica Italiana
\def\nphysa{Nucl.~Phys.~A}%
          % Nuclear Physics A
\def\physrep{Phys.~Rep.}%
          % Physics Reports
\def\physscr{Phys.~Scr}%
          % Physica Scripta
\def\planss{Planet.~Space~Sci.}%
          % Planetary Space Science
\def\procspie{Proc.~SPIE}%
          % Proceedings of the SPIE
\let\astap=\aap
\let\apjlett=\apjl
\let\apjsupp=\apjs
\let\applopt=\ao
%
%


\usepackage{natbib}
\bibliographystyle{abbrvnat}

\begin{document}

\title{\includegraphics[width=0.9\hsize]{ARCiS}\\User Guide to ARCiS}
\author{Michiel Min}
\date{\today}
\maketitle

\section{Introduction}


\subsection{Terms of use}

By using ARCiS you agree to the following:
\begin{itemize}
\item If in doubt on any of the results, you consult with me. Email: M.Min@sron.nl
\item You cite the appropriate papers listed below.
\end{itemize}
The most important reason for this is to make sure that ARCiS is used in a correct way and the result are scientifically useful. ARCiS is a complex code which can do a lot of things, this also means things can go wrong. Please refer to \cite{2020A&A...642A..28M} for the first full description of the fundamental properties of the code.

Note that there are several parts of the code from different developers:
\begin{itemize}
\item Cloud formation framework: \cite{2019A&A...622A.121O}
\item Optical properties of cloud particles computed using DHS: \cite{2005A&A...432..909M, 1981ApOpt..20.3657T}
\item Refractive indices for the cloud species, see references in \cite{2020A&A...642A..28M}
\item Molecular opacities: \cite{2021A&A...646A..21C} and references therein
\item Multinest Retrieval tools: \cite{2008MNRAS.384..449F, 2009MNRAS.398.1601F, 2019OJAp....2E..10F}
\item GGchem when including chemistry: \cite{2018A&A...614A...1W}
\item Disequilibrium chemistry implementation: \cite{2021A&A...656A..90K}
\item Diffusion implementation for 3D structures: \cite{2022A&A...665A...2C}
\item Coupling with planet formation parameters: \cite{2022A&A...667A.147K}
\end{itemize}

\section{Installing ARCiS}

Before installing ARCiS you need:
\begin{description}
\item{A Fortran compiler:} This can be either \texttt{gfortran} or \texttt{ifort} (any other might work but is not tested).
\item{\texttt{cfitsio} library:} This is needed to allow fortran to read and write binary fits files.
\item{\texttt{MultiNest}:} This allows ARCiS to perform Bayesian retrievals.
\end{description}

On a Mac the easiest is to install cfitsio through HomeBrew (google 'homebrew' for installation instructions. After that:
%
\shellcmd{brew install cfitsio}
%
Next fetch MultiNest:
%
\shellcmd{git clone https://github.com/JohannesBuchner/MultiNest.git}
\vspace{-2cm}\\
\shellcmd{cd MultiNest/build}
\vspace{-2cm}\\
\shellcmd{cmake ..}
\vspace{-2cm}\\
\shellcmd{make}
\vspace{-2cm}\\
\shellcmd{sudo make install}


Next fetch the git source code from:
%
\shellcmd{mkdir ARCiS ; cd ARCiS}
\vspace{-2cm}\\
\shellcmd{git clone https://github.com/michielmin/ARCiS.git ./src}
\vspace{-2cm}\\
\shellcmd{cd src}
\vspace{-2cm}\\
\shellcmd{make gfort=true multi=true}
%
This creates the ARCiS binary, which you can put in any path accessible (/usr/bin or something like that).

You also need data files to be stored under: \texttt{\$(HOME)/ARCiS/Data/}\\
These data files can be downloaded from \url{http://www.exoclouds.com} using the password: ARCiSData

\section{Using ARCiS}

To run ARCiS you further only need an input file. On the prompt type:
%
\shellcmd{ARCiS inputfile.dat -o outputdir}
%
which creates the output directory \texttt{outputdir} containing the output files.

There are several options included in ARCiS. These are given as keywords in the \texttt{inputfile.dat} file (or whatever you call it). Keywords are always given as \texttt{key=value} and can be anywhere in the file (order does not matter). Also, you can overwrite keywords set in the input file from the command line in the following way
%
\shellcmd{ARCiS inputfile.dat -o outputdir -s key1=value1}
%
Any number of keys can be set on the command line. Just make sure the first argument of the command line is the name of your input file. Note that ARCiS always takes the last keyword value it encounters, first reading the input file, next the command line keywords one by one.

\section{Keywords}

There are many keywords for a variety of different setups. Chemistry or not, cloud formation or not, various temperature structures, perform retrievals, 3D or 1D structure, and many more options. Below are some very basic setups discussed. The runtime of a single forward model can vary from 0.01 second to 15 minutes depending on your choices. For more information it is best to get in contact.

\subsection{Base properties}

\begin{description}
\item[\texttt{Rp}]
Radius of the planet in Jupiter radii.
\item[\texttt{Mp}]
Mass of the planet in Jupiter masses.
\item[\texttt{Pp}]
Atmospheric pressure corresponding to radius \texttt{Rp}. Default is 10\,bar.
\item[\texttt{Tstar}]
Temperature of the host star in K.
\item[\texttt{Rstar}]
Radius of the host star in Solar radii.
\item[\texttt{distance}]
distance to the system in parsec.
\item[\texttt{Dplanet}]
Distance of the planet to the star in AU.
\item[\texttt{planetname}]
Name of the planet to read from the database. Radius, mass and distance of the planet and the star are read from the database.
\end{description}

\subsection{Grid setup}

\begin{description}
\item[\texttt{pmin, pmax}]
Minimum, maximum pressure considered in the atmosphere
\item[\texttt{nr}]
Number of pressure points
\item[\texttt{lmin, lmax}]
Minimum, maximum wavelength considered in micron. Note that for temperature computations these must be set wide enough to ensure energy balance is properly computed.
\item[\texttt{specres}]
Spectral resolution R in lambda/dlambda
\item[\texttt{specresdust}]
Spectral resolution for computation of the solid state species in the clouds.
\end{description}

\subsection{Abundances of the molecules}

Homogeneous abundances can be set using keywords like e.g. \texttt{H2O=1d-4}. Only molecules that are defined through this somewhere in the input file are taken into account. These abundances are overwritten when chemistry is used.

\begin{description}
\item[\texttt{chemistry}]
Logical determining if chemistry is computed or not (either .true. or .false.)
\item[\texttt{condensates}]
Logical determining if condensates should be taken into account in the chemistry computations (default is .false. and most stable is to leave it like that).
\item[\texttt{COratio}]
C/O ratio of the atmosphere
\item[\texttt{metallicity}]
Metallicity of the atmosphere
\end{description}

\subsection{Opacities and raytracing}

\begin{description}
\item[\texttt{cia}]
Logical determining if CIA is taken into account
\item[\texttt{maxtau}]
Maximum optical depth considered for the raytracing
\item[\texttt{compute}]
Logical determining if the opacities need to be recomputed from the linelists
\item[\texttt{scattering}]
Logical determining if scattering of the thermal radiation is included
\item[\texttt{scattstar}]
Logical deternining if scattering from the star is included
\end{description}

\subsection{Temperature structure}

\begin{description}
\item[\texttt{computeT}]
Logical determining if the temperature structure is computed self-consistently
\item[\texttt{maxiter}]
Maximum number of iterations for the temperature structure
\item[\texttt{betaT}]
Cosine of the angle of incoming radiation.
\item[\texttt{TeffP}]
Effective temperature of the radiation from inside the planet
\item[\texttt{Tp}]
Temperature of the planet at 1 bar when \texttt{computeT=.false.}
\item[\texttt{dTp}]
Temperature gradient when \texttt{computeT=.false.}
\begin{equation}
\log_{10}(T[\mathrm{K}])=\log_{10}(T_p[\mathrm{K}])+dT_p\log_{10}(P[\mathrm{bar}])
\end{equation}
\end{description}

\subsection{Other options}

\subsubsection{$K_{zz}$ setup}
The diffusion of the atmosphere is important for the disequilibrium chemistry and for the cloud formation. For the clouds it is possible to set a separate homogeneous value or use the global defined $K_{zz}$ profile. The disequilibrium chemistry always uses the globally defined $K_{zz}$ profile.

Two cases are considered:\\
when $K_{zz}^{contrast} > 1$:
\begin{equation}
K_{zz}=\max\left[ K_{zz}^{deep} ; \min\left[ \left(K_{zz}^{deep}\cdot K_{zz}^{contrast} \right) ; \left(K_{zz}^{1bar} P^{-|\gamma_{K}|} \right) \right] \right]
\end{equation}
when $K_{zz}^{contrast} < 1$:
\begin{equation}
K_{zz}=\max\left[ \left(K_{zz}^{deep}\cdot K_{zz}^{contrast} \right) ; \min\left[ K_{zz}^{deep} ; \left(K_{zz}^{1bar} P^{|\gamma_{K}|} \right) \right] \right]
\end{equation}
This profile is used when $K_{zz}^{deep}$ and $K_{zz}^{1bar}$ are defined. By default these are set to negative values, which means the homogeneous value for $K_{zz}$ is used.

Keywords:
\begin{description}
\item[\texttt{Kzz}]
Sets the value used for homogeneous global $K_{zz}$
\item[\texttt{Kzz\_deep}]
Sets the value used for $K_{zz}^{deep}$
\item[\texttt{Kzz\_1bar}]
Sets the value used for $K_{zz}^{1bar}$
\item[\texttt{Kzz\_contrast}]
Sets the value used for $K_{zz}^{contrast}$
\item[\texttt{Kzz\_p}]
Sets the value of $\gamma_K$ (default is 0.5)
\end{description}

\subsubsection{Photochemical reactions}

There is the option to simulate photochemical reactions in ARCiS. This is done by converting molecules or atoms above a certain optical depth in the atmosphere into other molecules or hazes.
The conversion efficiency of the reaction is parameterised by the simple equation
\begin{equation}
f_{\rm conversion} = f_{\rm eff}\,e^{-\tau_{\rm UV}},
\end{equation}
Here $\tau_{\rm UV}$ is simply computed from the parameter $\kappa_{\rm UV}$ that can be given as an input parameter. The global efficiency $f_{\rm eff}$ determines the maximum efficiency possible. Note that $f_{\rm conversion}$ is capped at $1$ in the code. So it is possible to give $f_{\rm eff}>1$, but this will only result in an efficiency that is equal to $1$ up to lower pressures.

The global keyword is:
\begin{description}
\item[\texttt{kappaUV}]
Sets the value used for $\kappa_{\rm UV}$
\end{description}

The rest is set per reaction. 

As an example, consider we want to convert sulfur and oxygen into SO2
\begin{equation}
\rm S+2\,O \rightarrow SO_2
\end{equation}
This is done by setting
\begin{description}
\item[\texttt{photoreac1:S=1}]
Sets the reactant of reaction 1 to contain 1 S atom
\item[\texttt{photoreac1:O=2}]
Sets the reactant of reaction 1 to contain 2 O atoms
\item[\texttt{photoprod1:SO2=1}]
Sets the product of reaction 1 to contain 1 SO$_2$ molecule
\end{description}
Converting atoms to molecules is done before chemical calculations (so the atoms are removed from the atomic mixture and the molecular products are added later on)

We can also convert for example
\begin{equation}
\rm H_2S+2\,H_2O \rightarrow SO_2 +3\,H_2
\end{equation}
This is done by setting
\begin{description}
\item[\texttt{photoreac1:H2S=1}]
Sets the reactant of reaction 1 to contain 1 H2S molecule
\item[\texttt{photoreac1:H2O=2}]
Sets the reactant of reaction 1 to contain 2 H2O molecules
\item[\texttt{photoprod1:SO2=1}]
Sets the product of reaction 1 to contain 1 SO$_2$ molecule
\item[\texttt{photoprod1:H2=3}]
Sets the product of reaction 1 to contain 3 H$_2$ molecules
\end{description}
Converting molecules into other molecules is done after the chemical computations (so first it is determined how much H$_2$S and H$_2$O would be available.

If for a reaction the keyword
\begin{description}
\item[\texttt{photohaze1=.true.}]
\end{description}
is set. This means that if any mass remains after the conversion of reactants into products, this is converted into haze particles.

Finally the keyword
\begin{description}
\item[\texttt{photoeff1}] sets the value for $f_{\rm eff}$ of the reaction
\end{description}

\subsection{Retrieval}

\subsubsection{Observations}

\begin{description}
\item[\texttt{obs1:type}]
Can be "trans", "emis" or "emisR".
\item[\texttt{obs1:file}]
Filename with the observation. Should be in format:\\
column1: wavelength in micron\\
column2: trans or emis spectrum (same units as the output file to compare with)\\
column3: error\\
column4: spectral resolution of this wavelength bin (so $\lambda/\Delta\lambda$)
\item[\texttt{obs1:beta}]
Weight of this observation. Only relevant if more than one obs is defined.
\end{description}

\subsubsection{Parameters}

\begin{description}
\item[\texttt{fitpar:keyword}]
Keyword to be retrieved. This key switches automatically to the next retrieval parameter.
\item[\texttt{fitpar:min}]
Minimum value considered
\item[\texttt{fitpar:max}]
Maximum value considered
\item[\texttt{fitpar:log}]
Logical determining if the parameter is sampled logarithmically
\end{description}

\subsection{Very rough instrument simulation (use at own risk!)}

You can have ARCiS create simulated observations including estimate of the noise as function of wavelength. 
\begin{quote}
\emph{Note that this is absolutely not intended as a replacement of a proper instrument simulation! It only includes photon noise and can be used to get a very rough estimate of the expected performance.}
\end{quote}

\begin{description}

\item[\texttt{instrument1:name}]
Can be "MIRI", "NIRSPEC", "WFC3", "JWST" or "ARIEL". When it is something else it is assumed to be a filename containing a proper instrument simulation (file format currently the ExoSim format).
\item[\texttt{instrument1:ntrans}]
Number of transits to average over. When put to $0$, the number of transits is computed to assure that at each wavelength 7 scaleheights can be observed with $5\sigma$ accuracy.
\end{description}

\section{Output files}

There are many output files, most are for checking the model. The most important ones are discussed here.

\begin{description}
\item[\texttt{log.dat}]
This is the log file containing the runtime output.
\item[\texttt{input.dat}]
Copy of the input file used appended with the command line keywords. This allows rerunning exactly this model again without command line keywords.
\item[\texttt{mixingratios.dat}]
This file contains the temperature structure and the abundances of the molecules as a function of height in the atmosphere.
\item[\texttt{trans}]
Transmission spectrum. Header explains units.
\item[\texttt{emis}]
Emission spectrum. Header explains units.
\item[\texttt{emisR}]
Emission spectrum relative to the stellar emission.
\item[\texttt{.txt}]
File containing the MultiNest output when retrieval was done.
\item[\texttt{bestfit.dat}]
File containing the input file for the best fitting model when retrieval was done.
\end{description}

\section{Examples}

There are a few examples in the Example directory downloaded with the git repository for retrieval and forward modelling.

\bibliography{refs} % Entries are in the refs.bib file

\end{document}

